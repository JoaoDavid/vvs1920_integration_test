\documentclass[12pt]{article}
\usepackage[utf8]{inputenc} % input encoding
\usepackage[T1]{fontenc} % Use an 8-bit font encoding, so that ã is a
                         % single glyph in the font. Yields better
                         % hyphenation and better cut-and-paste from pdf
\usepackage{times} % Comment this line you you want the default font, Computer Roman
% \usepackage[portuguese]{babel} % Uncomment this file you plan to write in Portuguese
\usepackage{hyperref}
\usepackage{listings}
\usepackage{graphicx}
\usepackage[section]{placeins}
\usepackage{multicol}
\usepackage{tabularx}
\usepackage{amsmath}
\usepackage{tcolorbox}
\usepackage{physics}
\usepackage{hyperref}
\usepackage{float}
\graphicspath{ {../images/} }
\sloppy

\lstset{frame=tb,
  aboveskip=3mm,
  belowskip=3mm,
  showstringspaces=false,
  columns=flexible,
  basicstyle={\small\ttfamily},
  numbers=none,
  breaklines=true,
  breakatwhitespace=true,
  tabsize=3
}

\title{Assignment 2 Report \\
  \Large Software Verification and Validation \\ 2019--2020
}
\author{
  João David\\49448
}
\date{09/06/2020}

\begin{document}
\maketitle

\section{HTML Unit}
While defining the HTML Unit tests, some adaptations where made to the JSP files. The $name$ property was added to some HTML elements, in order to be able to get them using HTML Unit's API, those adaptations were made to the following files:

\begin{itemize}
   \item  CustomerInfo.jsp
   
   \item  SalesInfo.jsp
   
   \item  ShowSalesDelivery.jsp
   
   \item  addSaleDelivery.jsp  
\end{itemize}

\newpage

\section{DB Setup}
\subsection{Sales Behavior}
The first Junit test verifies that it is not possible to add a sale to an non existent customer. 
\begin{lstlisting}
public void extraSaleBehaviour1() throws ApplicationException {
	int vat = 503183504;
	assertFalse(hasClient(vat));
	assertThrows(ApplicationException.class, () -> {
		SaleService.INSTANCE.addSale(vat);
	});	
}	
\end{lstlisting}

All new sales created for a new customer must have its date the same as of creation, a total of 0.0, an open status and it needs to be associated to the right vat.
\begin{lstlisting}
public void extraSaleBehaviour2() throws ApplicationException {
	int vat = 503183504;
	assertFalse(hasClient(vat));
	CustomerService.INSTANCE.addCustomer(vat, "FCUL", 217500000);
	assertTrue(hasClient(vat));
	SaleService.INSTANCE.addSale(vat);
	List<SaleDTO> sales = SaleService.INSTANCE.getSaleByCustomerVat(vat).sales;
	SimpleDateFormat dateFormat = new SimpleDateFormat("yyyy-MM-dd");
	for (SaleDTO curr : sales) {
		assertEquals(dateFormat.format(new Date()), curr.data.toString());
		assertEquals(new Double(0.0), curr.total);
		assertEquals("O", curr.statusId);
		assertEquals(vat, curr.customerVat);
	}		
}
\end{lstlisting}
\subsection{Sale Deliveries Behavior}
After a sale has been closed, it should not be possible to add a delivery for that sale 
\begin{lstlisting}
public void extraSaleDeliveryBehaviour1() throws ApplicationException {
	int vat = 197672337;
	assertTrue(hasClient(vat));
	SaleService.INSTANCE.addSale(vat);
	assertEquals("O", SaleService.INSTANCE.getSaleById(1).statusId);
	SaleService.INSTANCE.updateSale(1);
	assertEquals("C", SaleService.INSTANCE.getSaleById(1).statusId);
	assertThrows(ApplicationException.class, () -> {
		SaleService.INSTANCE.addSaleDelivery(1, 1);
	});		
}
\end{lstlisting}

After removing a customer, its sale deliveries should be removed as well
\begin{lstlisting}
public void extraSaleDeliveryBehaviour2() throws ApplicationException {
	int vat = 197672337;
	assertTrue(hasClient(vat));
	SaleService.INSTANCE.addSaleDelivery(1, 1);
	assertNotEquals(0, SaleService.INSTANCE
			.getSalesDeliveryByVat(vat).sales_delivery.size());
	CustomerService.INSTANCE.removeCustomer(vat);
	assertFalse(hasClient(vat));
	assertEquals(0, SaleService.INSTANCE
		.getSalesDeliveryByVat(vat).sales_delivery.size());
}
\end{lstlisting}

\newpage
\section{Mockito}


\section{Bugs found}
\subsection{Customer Removal}
After removing a registered customer from the system, its addresses, sales and sale deliveries were still kept in the database.

\subsubsection{Reproduction}
\begin{enumerate}
   \item  Create e new customer
   
   \item  Add an address
   
   \item  Insert a new sale
   
   \item  Insert a new sale delivery using the previous two information
      
   \item  Remove the customer
      
   \item  Use the customer's vat number to search for sales/sale deliveries
\end{enumerate}


\subsubsection{Solution}
First implement in the Address, Sale and SaleDelivery RDGW classes, the methods responsible for deleting all addresses, sales and sale deliveries, respectively, given a customer VAT number. Then, use them to delete the information in the $removeCustomer$ method in the $CustomerService$ Java class.
\newpage
%------------------------------------------------------------------
\subsection{Insert new sales for non-existent customers}
The system allows the creation of new sales associated to VAT numbers that do not belong to any customer registered in the system.

\subsubsection{Reproduction}
\begin{enumerate}
   \item  Click on "Enter new sale"
   
   \item  Enter a VAT number that is not registered to any customer
\end{enumerate}


\subsubsection{Solution}
Within the $addSale$ method from the $SaleService$ class, use the method $getCustomerByVATNumber$ from the $CustomerRDGW$ class, to query the Customer's DB table, using his VAT number. If such customer does not exist, the system should raise an exception.

%------------------------------------------------------------------
\subsection{Adding an address}
The system allows the addition of addresses with invalid/empty parameters. For instance, one could add an address empty address or even an address where the zip code has no numbers, or a locality that is only numbers.

\subsubsection{Reproduction}
\begin{enumerate}
   \item  Click on "Insert new Address to Customer"
   
   \item  Enter a VAT number that is registered to a customer
   
   \item  Fill the fields with wrong values, or leave them empty
   
   \item  Insert the address
\end{enumerate}


\subsubsection{Solution}
Before inserting the address into the database, there should be a validation of the fields written by the user. There could also be a regular expression validation in the form itself.
\newpage
%------------------------------------------------------------------
\subsection{RDGW missing attribute initialization}
The constructor of $SaleDeliveryRowDataGateway$ is not initializing the attribute $address\_id$ and the constructor of $SaleRowDataGateway$ is not initializing the attributes $total$ and $statusId$. 


\subsubsection{Solution}
Using the ResultSet object passed to the constructor, initialize the previously mentioned attributes.
%------------------------------------------------------------------
\subsection{Creation of unnecessary objects}
The RDGW classes have empty constructors, that are being used to create RDGW objects in order to use their instance methods. This is a bad practice, this methods should be re factored to static, and the empty constructors removed. The methods in particular are getters, that return a RDGW object, or a list of RDGW's. 


\subsubsection{Solution}
Remove all RDGW empty constructors, then the Eclipse IDE will signal the places where they were previously used to invoke instance methods, change those methods to static, and then instead of using "new xRDGW().get..." use "xRDGW.get...".




\bibliographystyle{plain}
\end{document}

%%% Local Variables:
%%% mode: latex
%%% TeX-master: t
%%% End:
