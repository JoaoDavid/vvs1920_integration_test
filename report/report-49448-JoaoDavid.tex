\documentclass[12pt]{article}
\usepackage[utf8]{inputenc} % input encoding
\usepackage[T1]{fontenc} % Use an 8-bit font encoding, so that ã is a
                         % single glyph in the font. Yields better
                         % hyphenation and better cut-and-paste from pdf
\usepackage{times} % Comment this line you you want the default font, Computer Roman
% \usepackage[portuguese]{babel} % Uncomment this file you plan to write in Portuguese
\usepackage{hyperref}
\usepackage{listings}
\usepackage{graphicx}
\usepackage[section]{placeins}
\usepackage{multicol}
\usepackage{tabularx}
\usepackage{amsmath}
\usepackage{tcolorbox}
\usepackage{physics}
\usepackage{hyperref}
\usepackage{float}
\graphicspath{ {../images/} }
\sloppy

\lstset{frame=tb,
  aboveskip=3mm,
  belowskip=3mm,
  showstringspaces=false,
  columns=flexible,
  basicstyle={\small\ttfamily},
  numbers=none,
  breaklines=true,
  breakatwhitespace=true,
  tabsize=3
}

\title{Assignment 2 Report \\
  \Large Software Verification and Validation \\ 2019--2020
}
\author{
  João David\\49448
}
\date{09/06/2020}

\begin{document}
\maketitle

\section{HTML Unit}
While defining the HTML Unit tests, some adaptations where made to the JSP files. The $name$ property was added to some HTML elements, in order to be able to get them using HTML Unit's API, those adaptations were made to the following files:

\begin{itemize}
   \item  CustomerInfo.jsp
   
   \item  SalesInfo.jsp
   
   \item  ShowSalesDelivery.jsp
   
   \item  addSaleDelivery.jsp  
\end{itemize}

\newpage

\section{DB Setup}
\subsection{Sales Behavior}
The first Junit test verifies that it is not possible to add a sale to an non existent customer. 
\begin{lstlisting}
public void extraSaleBehaviour1() throws ApplicationException {
	int vat = 503183504;
	assertFalse(hasClient(vat));
	assertThrows(ApplicationException.class, () -> {
		SaleService.INSTANCE.addSale(vat);
	});	
}	
\end{lstlisting}

All new sales created for a new customer must have its date the same as of creation, a total of 0.0, an open status and it needs to be associated to the right vat.
\begin{lstlisting}
public void extraSaleBehaviour2() throws ApplicationException {
	int vat = 503183504;
	assertFalse(hasClient(vat));
	CustomerService.INSTANCE.addCustomer(vat, "FCUL", 217500000);
	assertTrue(hasClient(vat));
	SaleService.INSTANCE.addSale(vat);
	List<SaleDTO> sales = SaleService.INSTANCE.getSaleByCustomerVat(vat).sales;
	SimpleDateFormat dateFormat = new SimpleDateFormat("yyyy-MM-dd");
	for (SaleDTO curr : sales) {
		assertEquals(dateFormat.format(new Date()), curr.data.toString());
		assertEquals(new Double(0.0), curr.total);
		assertEquals("O", curr.statusId);
		assertEquals(vat, curr.customerVat);
	}		
}
\end{lstlisting}
\subsection{Sale Deliveries Behavior}

\section{Bugs found}









\bibliographystyle{plain}
\end{document}

%%% Local Variables:
%%% mode: latex
%%% TeX-master: t
%%% End:
